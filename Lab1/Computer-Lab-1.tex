\PassOptionsToPackage{unicode=true}{hyperref} % options for packages loaded elsewhere
\PassOptionsToPackage{hyphens}{url}
%
\documentclass[]{article}
\usepackage{lmodern}
\usepackage{amssymb,amsmath}
\usepackage{ifxetex,ifluatex}
\usepackage{fixltx2e} % provides \textsubscript
\ifnum 0\ifxetex 1\fi\ifluatex 1\fi=0 % if pdftex
  \usepackage[T1]{fontenc}
  \usepackage[utf8]{inputenc}
  \usepackage{textcomp} % provides euro and other symbols
\else % if luatex or xelatex
  \usepackage{unicode-math}
  \defaultfontfeatures{Ligatures=TeX,Scale=MatchLowercase}
\fi
% use upquote if available, for straight quotes in verbatim environments
\IfFileExists{upquote.sty}{\usepackage{upquote}}{}
% use microtype if available
\IfFileExists{microtype.sty}{%
\usepackage[]{microtype}
\UseMicrotypeSet[protrusion]{basicmath} % disable protrusion for tt fonts
}{}
\IfFileExists{parskip.sty}{%
\usepackage{parskip}
}{% else
\setlength{\parindent}{0pt}
\setlength{\parskip}{6pt plus 2pt minus 1pt}
}
\usepackage{hyperref}
\hypersetup{
            pdftitle={Computer Lab 1},
            pdfauthor={Dhyey Patel, Erik Anders},
            pdfborder={0 0 0},
            breaklinks=true}
\urlstyle{same}  % don't use monospace font for urls
\usepackage[margin=1in]{geometry}
\usepackage{color}
\usepackage{fancyvrb}
\newcommand{\VerbBar}{|}
\newcommand{\VERB}{\Verb[commandchars=\\\{\}]}
\DefineVerbatimEnvironment{Highlighting}{Verbatim}{commandchars=\\\{\}}
% Add ',fontsize=\small' for more characters per line
\usepackage{framed}
\definecolor{shadecolor}{RGB}{248,248,248}
\newenvironment{Shaded}{\begin{snugshade}}{\end{snugshade}}
\newcommand{\AlertTok}[1]{\textcolor[rgb]{0.94,0.16,0.16}{#1}}
\newcommand{\AnnotationTok}[1]{\textcolor[rgb]{0.56,0.35,0.01}{\textbf{\textit{#1}}}}
\newcommand{\AttributeTok}[1]{\textcolor[rgb]{0.77,0.63,0.00}{#1}}
\newcommand{\BaseNTok}[1]{\textcolor[rgb]{0.00,0.00,0.81}{#1}}
\newcommand{\BuiltInTok}[1]{#1}
\newcommand{\CharTok}[1]{\textcolor[rgb]{0.31,0.60,0.02}{#1}}
\newcommand{\CommentTok}[1]{\textcolor[rgb]{0.56,0.35,0.01}{\textit{#1}}}
\newcommand{\CommentVarTok}[1]{\textcolor[rgb]{0.56,0.35,0.01}{\textbf{\textit{#1}}}}
\newcommand{\ConstantTok}[1]{\textcolor[rgb]{0.00,0.00,0.00}{#1}}
\newcommand{\ControlFlowTok}[1]{\textcolor[rgb]{0.13,0.29,0.53}{\textbf{#1}}}
\newcommand{\DataTypeTok}[1]{\textcolor[rgb]{0.13,0.29,0.53}{#1}}
\newcommand{\DecValTok}[1]{\textcolor[rgb]{0.00,0.00,0.81}{#1}}
\newcommand{\DocumentationTok}[1]{\textcolor[rgb]{0.56,0.35,0.01}{\textbf{\textit{#1}}}}
\newcommand{\ErrorTok}[1]{\textcolor[rgb]{0.64,0.00,0.00}{\textbf{#1}}}
\newcommand{\ExtensionTok}[1]{#1}
\newcommand{\FloatTok}[1]{\textcolor[rgb]{0.00,0.00,0.81}{#1}}
\newcommand{\FunctionTok}[1]{\textcolor[rgb]{0.00,0.00,0.00}{#1}}
\newcommand{\ImportTok}[1]{#1}
\newcommand{\InformationTok}[1]{\textcolor[rgb]{0.56,0.35,0.01}{\textbf{\textit{#1}}}}
\newcommand{\KeywordTok}[1]{\textcolor[rgb]{0.13,0.29,0.53}{\textbf{#1}}}
\newcommand{\NormalTok}[1]{#1}
\newcommand{\OperatorTok}[1]{\textcolor[rgb]{0.81,0.36,0.00}{\textbf{#1}}}
\newcommand{\OtherTok}[1]{\textcolor[rgb]{0.56,0.35,0.01}{#1}}
\newcommand{\PreprocessorTok}[1]{\textcolor[rgb]{0.56,0.35,0.01}{\textit{#1}}}
\newcommand{\RegionMarkerTok}[1]{#1}
\newcommand{\SpecialCharTok}[1]{\textcolor[rgb]{0.00,0.00,0.00}{#1}}
\newcommand{\SpecialStringTok}[1]{\textcolor[rgb]{0.31,0.60,0.02}{#1}}
\newcommand{\StringTok}[1]{\textcolor[rgb]{0.31,0.60,0.02}{#1}}
\newcommand{\VariableTok}[1]{\textcolor[rgb]{0.00,0.00,0.00}{#1}}
\newcommand{\VerbatimStringTok}[1]{\textcolor[rgb]{0.31,0.60,0.02}{#1}}
\newcommand{\WarningTok}[1]{\textcolor[rgb]{0.56,0.35,0.01}{\textbf{\textit{#1}}}}
\usepackage{graphicx,grffile}
\makeatletter
\def\maxwidth{\ifdim\Gin@nat@width>\linewidth\linewidth\else\Gin@nat@width\fi}
\def\maxheight{\ifdim\Gin@nat@height>\textheight\textheight\else\Gin@nat@height\fi}
\makeatother
% Scale images if necessary, so that they will not overflow the page
% margins by default, and it is still possible to overwrite the defaults
% using explicit options in \includegraphics[width, height, ...]{}
\setkeys{Gin}{width=\maxwidth,height=\maxheight,keepaspectratio}
\setlength{\emergencystretch}{3em}  % prevent overfull lines
\providecommand{\tightlist}{%
  \setlength{\itemsep}{0pt}\setlength{\parskip}{0pt}}
\setcounter{secnumdepth}{0}
% Redefines (sub)paragraphs to behave more like sections
\ifx\paragraph\undefined\else
\let\oldparagraph\paragraph
\renewcommand{\paragraph}[1]{\oldparagraph{#1}\mbox{}}
\fi
\ifx\subparagraph\undefined\else
\let\oldsubparagraph\subparagraph
\renewcommand{\subparagraph}[1]{\oldsubparagraph{#1}\mbox{}}
\fi

% set default figure placement to htbp
\makeatletter
\def\fps@figure{htbp}
\makeatother


\title{Computer Lab 1}
\author{Dhyey Patel, Erik Anders}
\date{4/18/2020}

\begin{document}
\maketitle

\#\#1. Bernoulli \ldots{} again. Let y 1 , \ldots{}, y n \textbar{}θ ∼
Bern(θ), and assume that you have obtained a sample with s = 5 successes
in n = 20 trials. Assume a Beta(α 0 , β 0 ) prior for θ and let α 0 = β
0 = 2.

\begin{enumerate}
\def\labelenumi{(\alph{enumi})}
\tightlist
\item
  Draw random numbers from the posterior θ\textbar{}y ∼ Beta(α 0 + s, β
  0 + f ), y = (y 1 , . . . , y n ), and verify graphically that the
  posterior mean and standard de- viation converges to the true values
  as the number of random draws grows large.
\end{enumerate}

\begin{verbatim}
## [1] "True mean"
\end{verbatim}

\begin{verbatim}
## [1] 0.2916667
\end{verbatim}

\begin{verbatim}
## [1] "True sd"
\end{verbatim}

\begin{verbatim}
## [1] 0.008263889
\end{verbatim}

\includegraphics{Computer-Lab-1_files/figure-latex/unnamed-chunk-1-1.pdf}
As shown above, the mean (black) and the standard deviation (red)
converge very quickly to their true values.

\includegraphics{Computer-Lab-1_files/figure-latex/unnamed-chunk-2-1.pdf}
\includegraphics{Computer-Lab-1_files/figure-latex/unnamed-chunk-2-2.pdf}
\includegraphics{Computer-Lab-1_files/figure-latex/unnamed-chunk-2-3.pdf}
\includegraphics{Computer-Lab-1_files/figure-latex/unnamed-chunk-2-4.pdf}

As we can see from the plots, as n increases from 50 to 1000, the mean
visually converges to the true mean 0.29.

\begin{enumerate}
\def\labelenumi{(\alph{enumi})}
\setcounter{enumi}{1}
\tightlist
\item
  Use simulation (nDraws = 10000) to compute the posterior probability
  Pr(θ \textgreater{} 0.3\textbar{}y) and compare with the exact value
  {[}Hint: pbeta(){]}. θ by simulation
\end{enumerate}

\begin{verbatim}
## [1] "simulated probability"
\end{verbatim}

\begin{verbatim}
## [1] 0.4406
\end{verbatim}

\begin{verbatim}
## [1] "exact probability"
\end{verbatim}

\begin{verbatim}
## [1] 0.4399472
\end{verbatim}

Posterior probability by simulation comes up to be approximately 0.439.
Posterior probability theoretically is 0.4399.

\begin{enumerate}
\def\labelenumi{(\alph{enumi})}
\setcounter{enumi}{2}
\tightlist
\item
  Compute the posterior distribution of the log-odds φ = log 1−θ (nDraws
  = 10000). {[}Hint: hist() and density() might come in handy{]}
\end{enumerate}

\includegraphics{Computer-Lab-1_files/figure-latex/unnamed-chunk-4-1.pdf}

\begin{verbatim}
## 
## Call:
##  density.default(x = log_odds)
## 
## Data: log_odds (10000 obs.); Bandwidth 'bw' = 0.06534
## 
##        x                  y            
##  Min.   :-3.07403   Min.   :0.0000069  
##  1st Qu.:-2.05509   1st Qu.:0.0037942  
##  Median :-1.03615   Median :0.0752536  
##  Mean   :-1.03615   Mean   :0.2451139  
##  3rd Qu.:-0.01722   3rd Qu.:0.4592289  
##  Max.   : 1.00172   Max.   :0.8666984
\end{verbatim}

\#\#2. Log-normal distribution and the Gini coefficient. Assume that you
have asked 10 randomly selected persons about their monthly in- come (in
thousands Swedish Krona) and obtained the following ten observations:
44, 25, 45, 52, 30, 63, 19, 50, 34 and 67. A common model for
non-negative continuous variables is the log-normal distribution. The
log-normal distribution log N (μ, σ 2 ) has density function

\[p(y| \mu, \sigma²) = \frac{1}{y*\sqrt{2\pi\sigma²}}\exp(-\frac{1}{2\sigma²}(\log y-\mu)²),\]

for y \textgreater{} 0, μ \textgreater{} 0 and σ 2 \textgreater{} 0. The
log-normal distribution is related to the normal distribution as
follows: if y ∼ log N (μ, σ 2 ) then log y ∼ N (μ, σ 2 ). Let iid y 1 ,
\ldots{}, y n \textbar{}μ, σ 2 ∼ log N (μ, σ 2 ), where μ = 3.7 is
assumed to be known but σ 2 is unknown with non-informative prior p(σ 2
) ∝ 1/σ 2 . The posterior for σ 2 is the Inv − χ 2 (n, τ 2 )
distribution, where

\[\tau² = \frac{\sum_{i=1}^{n}(\log y_i-\mu)²}{n}\]

\begin{enumerate}
\def\labelenumi{(\alph{enumi})}
\tightlist
\item
  Simulate 10, 000 draws from the posterior of σ 2 (assuming μ = 3.7)
  and com- pare with the theoretical Inv − χ 2 (n, τ 2 ) posterior
  distribution.
\end{enumerate}

\includegraphics{Computer-Lab-1_files/figure-latex/unnamed-chunk-5-1.pdf}

\begin{verbatim}
## [1] "Simulated mean"
\end{verbatim}

\begin{verbatim}
## [1] 0.1869669
\end{verbatim}

\begin{verbatim}
## [1] "Simulated variance"
\end{verbatim}

\begin{verbatim}
## [1] 0.01215595
\end{verbatim}

\begin{verbatim}
## [1] "Theoretical mean"
\end{verbatim}

\begin{verbatim}
## [1] 0.1874264
\end{verbatim}

\begin{verbatim}
## [1] "Theoretical variance"
\end{verbatim}

\begin{verbatim}
## [1] 0.01170956
\end{verbatim}

To compare the simulated and theoretical distribution we chose the mean
and standard deviation. As shown in the output above, they are both very
close to each other.

\begin{enumerate}
\def\labelenumi{(\alph{enumi})}
\setcounter{enumi}{1}
\tightlist
\item
  The most common measure of income inequality is the Gini coefficient,
  G, where 0 ≤ G ≤ 1. G = 0 means a completely equal income
  distribution, whereas G = 1 means complete income inequality. √  See
  Wikipedia for more information. It can be shown that G = 2Φ σ/ 2 − 1
  when incomes follow a log N (μ, σ 2 ) distribution. Φ(z) is the
  cumulative distribution function (CDF) for the standard normal
  distribution with mean zero and unit variance. Use the posterior draws
  in a) to compute the posterior distribution of the Gini coefficient G
  for the current data set.
\end{enumerate}

\includegraphics{Computer-Lab-1_files/figure-latex/unnamed-chunk-6-1.pdf}

\begin{enumerate}
\def\labelenumi{(\alph{enumi})}
\setcounter{enumi}{2}
\tightlist
\item
  Use the posterior draws from b) to compute a 90\% equal tail credible
  interval for G. A 90\% equal tail interval (a, b) cuts off 5\% percent
  of the posterior probability mass to the left of a, and 5\% to the
  right of b. Also, do a kernel density estimate of the posterior of G
  using the density function in R with default settings, and use that
  kernel density estimate to compute a 90\% Highest Posterior Density
  interval for G. Compare the two intervals.
\end{enumerate}

\includegraphics{Computer-Lab-1_files/figure-latex/unnamed-chunk-7-1.pdf}
\#\# still need to compare

\#\#3. Bayesian inference for the concentration parameter in the von
Mises distribution. This exercise is concerned with directional data.
The point is to show you that the posterior distribution for somewhat
weird models can be obtained by plotting it over a grid of values. The
data points are observed wind directions at a given location on ten
different days. The data are recorded in degrees: (40, 303, 326, 285,
296, 314, 20, 308, 299, 296), where North is located at zero degrees
(see Figure 1 on the next page, where the angles are measured
clockwise). To fit with Wikipedias description of probability
distributions for circular data we convert the data into radians −π ≤ y
≤ π . The 10 observations in radians are (−2.44, 2.14, 2.54, 1.83, 2.02,
2.33, −2.79, 2.23, 2.07, 2.02). Assume that these data points are
independent observations following the von Mises distribution

\[p(y|\mu,\kappa) = \frac{exp(\kappa*\cos(y-\mu))}{2\pi*I_0(\kappa)}, -\pi \leq y \leq +\pi,\]

where I 0 (κ) is the modified Bessel function of the first kind of order
zero {[}see ?besselI in R{]}. The parameter μ (−π ≤ μ ≤ π) is the mean
direction and κ \textgreater{} 0 is called the concentration parameter.
Large κ gives a small variance around μ, and vice versa. Assume that μ
is known to be 2.39. Let κ ∼ Exponential(λ = 1) a priori, where λ is the
rate parameter of the exponential distribution (so that the mean is
1/λ). (a) Plot the posterior distribution of κ for the wind direction
data over a fine grid of κ values.

\includegraphics{Computer-Lab-1_files/figure-latex/unnamed-chunk-8-1.pdf}

\begin{enumerate}
\def\labelenumi{(\alph{enumi})}
\setcounter{enumi}{1}
\tightlist
\item
  Find the (approximate) posterior mode of κ from the information in a).
\end{enumerate}

\includegraphics{Computer-Lab-1_files/figure-latex/unnamed-chunk-9-1.pdf}

\#\#Code Appendix

\begin{Shaded}
\begin{Highlighting}[]
\NormalTok{knitr}\OperatorTok{::}\NormalTok{opts_chunk}\OperatorTok{$}\KeywordTok{set}\NormalTok{(}\DataTypeTok{echo =} \OtherTok{TRUE}\NormalTok{)}
\CommentTok{#1.}
\CommentTok{#a).}

\NormalTok{avg <-}\StringTok{ }\KeywordTok{c}\NormalTok{()}
\NormalTok{sdv <-}\StringTok{ }\KeywordTok{c}\NormalTok{()}
\ControlFlowTok{for}\NormalTok{ (n }\ControlFlowTok{in} \KeywordTok{c}\NormalTok{(}\DecValTok{1}\NormalTok{,}\DecValTok{10}\NormalTok{,}\DecValTok{100}\NormalTok{,}\DecValTok{1000}\NormalTok{,}\DecValTok{10000}\NormalTok{)) \{}
\NormalTok{  dist <-}\StringTok{ }\KeywordTok{rbeta}\NormalTok{(}\DataTypeTok{n =}\NormalTok{ n,}\DataTypeTok{shape1 =} \DecValTok{2}\OperatorTok{+}\DecValTok{5}\NormalTok{,}\DataTypeTok{shape2 =} \DecValTok{2}\OperatorTok{+}\DecValTok{15}\NormalTok{)}
  \CommentTok{#plot(dist)}
\NormalTok{  avg <-}\StringTok{ }\KeywordTok{c}\NormalTok{(avg,}\KeywordTok{mean}\NormalTok{(dist))}
\NormalTok{  sdv <-}\StringTok{ }\KeywordTok{c}\NormalTok{(sdv, }\KeywordTok{sd}\NormalTok{(dist))}
\NormalTok{\}}

\NormalTok{a =}\StringTok{ }\DecValTok{7}
\NormalTok{b =}\StringTok{ }\DecValTok{17}
\NormalTok{mean_th =}\StringTok{ }\NormalTok{a}\OperatorTok{/}\NormalTok{(a}\OperatorTok{+}\NormalTok{b)}
\NormalTok{sd_th =}\StringTok{ }\NormalTok{a}\OperatorTok{*}\NormalTok{b}\OperatorTok{/}\NormalTok{(((a}\OperatorTok{+}\NormalTok{b)}\OperatorTok{^}\DecValTok{2}\NormalTok{) }\OperatorTok{*}\StringTok{ }\NormalTok{(a}\OperatorTok{+}\NormalTok{b}\OperatorTok{+}\DecValTok{1}\NormalTok{)) }

\KeywordTok{print}\NormalTok{(}\StringTok{"True mean"}\NormalTok{)}
\KeywordTok{print}\NormalTok{(mean_th)}
\KeywordTok{print}\NormalTok{(}\StringTok{"True sd"}\NormalTok{)}
\KeywordTok{print}\NormalTok{(sd_th)}

\CommentTok{#E = (alpha + s)/(alpha + s +beta+ f)}
\CommentTok{# = (alpha + 5)/(alpha + 5+beta + 15)}
\CommentTok{# = (alpha + 5)/(alpha + beta + 20)}
\CommentTok{# = 7/24 = 0.2916667}

\KeywordTok{plot}\NormalTok{(}\KeywordTok{c}\NormalTok{(}\DecValTok{1}\NormalTok{,}\DecValTok{10}\NormalTok{,}\DecValTok{100}\NormalTok{,}\DecValTok{1000}\NormalTok{,}\DecValTok{10000}\NormalTok{),avg, }\DataTypeTok{type =} \StringTok{"l"}\NormalTok{, }\DataTypeTok{ylim =} \KeywordTok{c}\NormalTok{(}\DecValTok{0}\NormalTok{,}\FloatTok{0.4}\NormalTok{), }\DataTypeTok{ylab =} \StringTok{"sd, mean"}\NormalTok{)}
\KeywordTok{lines}\NormalTok{(}\KeywordTok{c}\NormalTok{(}\DecValTok{1}\NormalTok{,}\DecValTok{10}\NormalTok{,}\DecValTok{100}\NormalTok{,}\DecValTok{1000}\NormalTok{,}\DecValTok{10000}\NormalTok{),sdv, }\DataTypeTok{col=}\StringTok{"red"}\NormalTok{)}

\KeywordTok{hist}\NormalTok{(}\KeywordTok{rbeta}\NormalTok{(}\DecValTok{50}\NormalTok{,a,b),}\DataTypeTok{freq =} \OtherTok{FALSE}\NormalTok{, }\DataTypeTok{breaks =} \DecValTok{20}\NormalTok{)}

\KeywordTok{hist}\NormalTok{(}\KeywordTok{rbeta}\NormalTok{(}\DecValTok{100}\NormalTok{,a,b),}\DataTypeTok{freq =} \OtherTok{FALSE}\NormalTok{, }\DataTypeTok{breaks =} \DecValTok{20}\NormalTok{)}

\KeywordTok{hist}\NormalTok{(}\KeywordTok{rbeta}\NormalTok{(}\DecValTok{500}\NormalTok{,a,b),}\DataTypeTok{freq =} \OtherTok{FALSE}\NormalTok{, }\DataTypeTok{breaks =} \DecValTok{20}\NormalTok{)}

\KeywordTok{hist}\NormalTok{(}\KeywordTok{rbeta}\NormalTok{(}\DecValTok{1000}\NormalTok{,a,b),}\DataTypeTok{freq =} \OtherTok{FALSE}\NormalTok{, }\DataTypeTok{breaks =} \DecValTok{20}\NormalTok{)}

\CommentTok{#b).}
\NormalTok{dist <-}\KeywordTok{rbeta}\NormalTok{(}\DataTypeTok{n =} \DecValTok{10000}\NormalTok{,}\DataTypeTok{shape1 =} \DecValTok{2}\OperatorTok{+}\DecValTok{5}\NormalTok{,}\DataTypeTok{shape2 =} \DecValTok{2}\OperatorTok{+}\DecValTok{15}\NormalTok{)}
\KeywordTok{print}\NormalTok{(}\StringTok{"simulated probability"}\NormalTok{)}
\KeywordTok{sum}\NormalTok{(dist}\OperatorTok{>}\FloatTok{0.3}\NormalTok{)}\OperatorTok{/}\KeywordTok{length}\NormalTok{(dist)}
\KeywordTok{print}\NormalTok{(}\StringTok{"exact probability"}\NormalTok{)}
\DecValTok{1}\OperatorTok{-}\KeywordTok{pbeta}\NormalTok{(}\FloatTok{0.3}\NormalTok{,}\DataTypeTok{shape1 =} \DecValTok{2}\OperatorTok{+}\DecValTok{5}\NormalTok{,}\DataTypeTok{shape2 =} \DecValTok{2}\OperatorTok{+}\DecValTok{15}\NormalTok{)}

\CommentTok{#c).}
\NormalTok{log_odds<-}\KeywordTok{log}\NormalTok{(dist}\OperatorTok{/}\NormalTok{(}\DecValTok{1}\OperatorTok{-}\NormalTok{dist))}
\KeywordTok{hist}\NormalTok{(log_odds)}
\KeywordTok{density}\NormalTok{(log_odds)}

\CommentTok{#2.}
\CommentTok{#a).}
\NormalTok{obs <-}\StringTok{ }\KeywordTok{c}\NormalTok{(}\DecValTok{44}\NormalTok{,}\DecValTok{25}\NormalTok{,}\DecValTok{45}\NormalTok{,}\DecValTok{52}\NormalTok{,}\DecValTok{30}\NormalTok{,}\DecValTok{63}\NormalTok{,}\DecValTok{19}\NormalTok{,}\DecValTok{50}\NormalTok{,}\DecValTok{34}\NormalTok{,}\DecValTok{67}\NormalTok{)}

\NormalTok{n <-}\StringTok{ }\DecValTok{10000}
\NormalTok{tau2 <-}\StringTok{ }\KeywordTok{sum}\NormalTok{((}\KeywordTok{log}\NormalTok{(obs)}\OperatorTok{-}\FloatTok{3.7}\NormalTok{)}\OperatorTok{^}\DecValTok{2}\NormalTok{)}\OperatorTok{/}\NormalTok{(}\KeywordTok{length}\NormalTok{(obs))}
\NormalTok{posterior <-}\StringTok{ }\NormalTok{((}\KeywordTok{length}\NormalTok{(obs))}\OperatorTok{*}\NormalTok{tau2)}\OperatorTok{/}\KeywordTok{rchisq}\NormalTok{(}\DataTypeTok{n =}\NormalTok{ n, }\DataTypeTok{df =} \KeywordTok{length}\NormalTok{(obs))}
\KeywordTok{hist}\NormalTok{(posterior)}

\CommentTok{#We compare mean of simulated posterior and theoretical value}
\CommentTok{#simulated:}
\KeywordTok{print}\NormalTok{(}\StringTok{"Simulated mean"}\NormalTok{)}
\KeywordTok{mean}\NormalTok{(posterior)}
\KeywordTok{print}\NormalTok{(}\StringTok{"Simulated variance"}\NormalTok{)}
\KeywordTok{var}\NormalTok{(posterior)}

\CommentTok{#theoretical:}
\KeywordTok{print}\NormalTok{(}\StringTok{"Theoretical mean"}\NormalTok{)}
\NormalTok{(}\KeywordTok{length}\NormalTok{(obs)}\OperatorTok{*}\NormalTok{tau2)}\OperatorTok{/}\NormalTok{(}\KeywordTok{length}\NormalTok{(obs)}\OperatorTok{-}\DecValTok{2}\NormalTok{)}
\KeywordTok{print}\NormalTok{(}\StringTok{"Theoretical variance"}\NormalTok{)}
\NormalTok{(}\DecValTok{2}\OperatorTok{*}\KeywordTok{length}\NormalTok{(obs)}\OperatorTok{^}\DecValTok{2}\OperatorTok{*}\NormalTok{tau2}\OperatorTok{^}\DecValTok{2}\NormalTok{)}\OperatorTok{/}\NormalTok{(((}\KeywordTok{length}\NormalTok{(obs)}\OperatorTok{-}\DecValTok{2}\NormalTok{)}\OperatorTok{^}\DecValTok{2}\NormalTok{) }\OperatorTok{*}\StringTok{ }\NormalTok{(}\KeywordTok{length}\NormalTok{(obs)}\OperatorTok{-}\DecValTok{4}\NormalTok{))}

\CommentTok{#b).}
\NormalTok{G <-}\StringTok{ }\DecValTok{2}\OperatorTok{*}\StringTok{ }\KeywordTok{pnorm}\NormalTok{(}\DataTypeTok{q =} \KeywordTok{sqrt}\NormalTok{(posterior)}\OperatorTok{/}\KeywordTok{sqrt}\NormalTok{(}\DecValTok{2}\NormalTok{), }\DataTypeTok{mean =} \DecValTok{0}\NormalTok{, }\DataTypeTok{sd =} \DecValTok{1}\NormalTok{)}\OperatorTok{-}\DecValTok{1}
\KeywordTok{hist}\NormalTok{(G, }\DataTypeTok{breaks =} \DecValTok{30}\NormalTok{)}

\CommentTok{#c).}
\CommentTok{#90% equal tail credible interval}
\NormalTok{G_sort <-}\StringTok{ }\KeywordTok{sort}\NormalTok{(G)}
\NormalTok{G_cut <-}\StringTok{ }\NormalTok{G_sort[(}\KeywordTok{length}\NormalTok{(G_sort)}\OperatorTok{*}\FloatTok{0.05}\NormalTok{)}\OperatorTok{:}\NormalTok{(}\KeywordTok{length}\NormalTok{(G_sort)}\OperatorTok{*}\FloatTok{0.95}\DecValTok{-1}\NormalTok{)]}
\KeywordTok{hist}\NormalTok{(G, }\DataTypeTok{breaks =} \DecValTok{30}\NormalTok{)}
\KeywordTok{abline}\NormalTok{(}\DataTypeTok{v=}\KeywordTok{min}\NormalTok{(G_cut), }\DataTypeTok{col =} \StringTok{"red"}\NormalTok{)}
\KeywordTok{abline}\NormalTok{(}\DataTypeTok{v=}\KeywordTok{max}\NormalTok{(G_cut), }\DataTypeTok{col =} \StringTok{"red"}\NormalTok{)}

\CommentTok{#3.}
\CommentTok{#a).}
\NormalTok{wind_dir <-}\StringTok{ }\KeywordTok{c}\NormalTok{(}\DecValTok{40}\NormalTok{,}\DecValTok{303}\NormalTok{,}\DecValTok{326}\NormalTok{,}\DecValTok{285}\NormalTok{,}\DecValTok{296}\NormalTok{,}\DecValTok{314}\NormalTok{,}\DecValTok{20}\NormalTok{,}\DecValTok{308}\NormalTok{,}\DecValTok{299}\NormalTok{,}\DecValTok{296}\NormalTok{)}
\NormalTok{radians <-}\StringTok{ }\KeywordTok{c}\NormalTok{(}\OperatorTok{-}\FloatTok{2.44}\NormalTok{,}\FloatTok{2.14}\NormalTok{,}\FloatTok{2.54}\NormalTok{,}\FloatTok{1.83}\NormalTok{,}\FloatTok{2.02}\NormalTok{,}\FloatTok{2.33}\NormalTok{,}\OperatorTok{-}\FloatTok{2.79}\NormalTok{,}\FloatTok{2.23}\NormalTok{,}\FloatTok{2.07}\NormalTok{,}\FloatTok{2.02}\NormalTok{)}

\NormalTok{mu <-}\StringTok{ }\FloatTok{2.39}
\NormalTok{k <-}\StringTok{ }\KeywordTok{seq}\NormalTok{(}\FloatTok{0.001}\NormalTok{, }\FloatTok{10.001}\NormalTok{, }\DataTypeTok{by =} \FloatTok{0.001}\NormalTok{)}
\NormalTok{n <-}\StringTok{ }\KeywordTok{length}\NormalTok{(radians)}
\NormalTok{ml <-}\StringTok{ }\KeywordTok{exp}\NormalTok{(k}\OperatorTok{*}\KeywordTok{sum}\NormalTok{(}\KeywordTok{cos}\NormalTok{(radians}\OperatorTok{-}\NormalTok{mu)))}\OperatorTok{/}\NormalTok{(}\KeywordTok{besselI}\NormalTok{(k, }\DecValTok{0}\NormalTok{))}\OperatorTok{^}\NormalTok{n}
\NormalTok{prior <-}\StringTok{ }\KeywordTok{exp}\NormalTok{(}\OperatorTok{-}\NormalTok{k)}
\CommentTok{#posterior <- exp(k*sum(cos(radians-mu))-k)/(besselI(k, 0))^n}
\NormalTok{posterior <-}\StringTok{ }\NormalTok{ml }\OperatorTok{*}\StringTok{ }\NormalTok{prior }\CommentTok{#distribution the same but values very small}
\KeywordTok{plot}\NormalTok{(k,posterior, }\DataTypeTok{type =} \StringTok{"l"}\NormalTok{)}

\CommentTok{#b).}
\KeywordTok{plot}\NormalTok{(k,posterior, }\DataTypeTok{type =} \StringTok{"l"}\NormalTok{)}
\KeywordTok{abline}\NormalTok{(}\DataTypeTok{v=}\NormalTok{k[}\KeywordTok{which}\NormalTok{(posterior}\OperatorTok{==}\KeywordTok{max}\NormalTok{(posterior))], }\DataTypeTok{col =} \StringTok{"red"}\NormalTok{)}
\end{Highlighting}
\end{Shaded}

\end{document}
