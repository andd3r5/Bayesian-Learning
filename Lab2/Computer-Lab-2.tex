\PassOptionsToPackage{unicode=true}{hyperref} % options for packages loaded elsewhere
\PassOptionsToPackage{hyphens}{url}
%
\documentclass[]{article}
\usepackage{lmodern}
\usepackage{amssymb,amsmath}
\usepackage{ifxetex,ifluatex}
\usepackage{fixltx2e} % provides \textsubscript
\ifnum 0\ifxetex 1\fi\ifluatex 1\fi=0 % if pdftex
  \usepackage[T1]{fontenc}
  \usepackage[utf8]{inputenc}
  \usepackage{textcomp} % provides euro and other symbols
\else % if luatex or xelatex
  \usepackage{unicode-math}
  \defaultfontfeatures{Ligatures=TeX,Scale=MatchLowercase}
\fi
% use upquote if available, for straight quotes in verbatim environments
\IfFileExists{upquote.sty}{\usepackage{upquote}}{}
% use microtype if available
\IfFileExists{microtype.sty}{%
\usepackage[]{microtype}
\UseMicrotypeSet[protrusion]{basicmath} % disable protrusion for tt fonts
}{}
\IfFileExists{parskip.sty}{%
\usepackage{parskip}
}{% else
\setlength{\parindent}{0pt}
\setlength{\parskip}{6pt plus 2pt minus 1pt}
}
\usepackage{hyperref}
\hypersetup{
            pdftitle={Computer Lab 2},
            pdfauthor={Dhyey Patel, Erik Anders},
            pdfborder={0 0 0},
            breaklinks=true}
\urlstyle{same}  % don't use monospace font for urls
\usepackage[margin=1in]{geometry}
\usepackage{color}
\usepackage{fancyvrb}
\newcommand{\VerbBar}{|}
\newcommand{\VERB}{\Verb[commandchars=\\\{\}]}
\DefineVerbatimEnvironment{Highlighting}{Verbatim}{commandchars=\\\{\}}
% Add ',fontsize=\small' for more characters per line
\usepackage{framed}
\definecolor{shadecolor}{RGB}{248,248,248}
\newenvironment{Shaded}{\begin{snugshade}}{\end{snugshade}}
\newcommand{\AlertTok}[1]{\textcolor[rgb]{0.94,0.16,0.16}{#1}}
\newcommand{\AnnotationTok}[1]{\textcolor[rgb]{0.56,0.35,0.01}{\textbf{\textit{#1}}}}
\newcommand{\AttributeTok}[1]{\textcolor[rgb]{0.77,0.63,0.00}{#1}}
\newcommand{\BaseNTok}[1]{\textcolor[rgb]{0.00,0.00,0.81}{#1}}
\newcommand{\BuiltInTok}[1]{#1}
\newcommand{\CharTok}[1]{\textcolor[rgb]{0.31,0.60,0.02}{#1}}
\newcommand{\CommentTok}[1]{\textcolor[rgb]{0.56,0.35,0.01}{\textit{#1}}}
\newcommand{\CommentVarTok}[1]{\textcolor[rgb]{0.56,0.35,0.01}{\textbf{\textit{#1}}}}
\newcommand{\ConstantTok}[1]{\textcolor[rgb]{0.00,0.00,0.00}{#1}}
\newcommand{\ControlFlowTok}[1]{\textcolor[rgb]{0.13,0.29,0.53}{\textbf{#1}}}
\newcommand{\DataTypeTok}[1]{\textcolor[rgb]{0.13,0.29,0.53}{#1}}
\newcommand{\DecValTok}[1]{\textcolor[rgb]{0.00,0.00,0.81}{#1}}
\newcommand{\DocumentationTok}[1]{\textcolor[rgb]{0.56,0.35,0.01}{\textbf{\textit{#1}}}}
\newcommand{\ErrorTok}[1]{\textcolor[rgb]{0.64,0.00,0.00}{\textbf{#1}}}
\newcommand{\ExtensionTok}[1]{#1}
\newcommand{\FloatTok}[1]{\textcolor[rgb]{0.00,0.00,0.81}{#1}}
\newcommand{\FunctionTok}[1]{\textcolor[rgb]{0.00,0.00,0.00}{#1}}
\newcommand{\ImportTok}[1]{#1}
\newcommand{\InformationTok}[1]{\textcolor[rgb]{0.56,0.35,0.01}{\textbf{\textit{#1}}}}
\newcommand{\KeywordTok}[1]{\textcolor[rgb]{0.13,0.29,0.53}{\textbf{#1}}}
\newcommand{\NormalTok}[1]{#1}
\newcommand{\OperatorTok}[1]{\textcolor[rgb]{0.81,0.36,0.00}{\textbf{#1}}}
\newcommand{\OtherTok}[1]{\textcolor[rgb]{0.56,0.35,0.01}{#1}}
\newcommand{\PreprocessorTok}[1]{\textcolor[rgb]{0.56,0.35,0.01}{\textit{#1}}}
\newcommand{\RegionMarkerTok}[1]{#1}
\newcommand{\SpecialCharTok}[1]{\textcolor[rgb]{0.00,0.00,0.00}{#1}}
\newcommand{\SpecialStringTok}[1]{\textcolor[rgb]{0.31,0.60,0.02}{#1}}
\newcommand{\StringTok}[1]{\textcolor[rgb]{0.31,0.60,0.02}{#1}}
\newcommand{\VariableTok}[1]{\textcolor[rgb]{0.00,0.00,0.00}{#1}}
\newcommand{\VerbatimStringTok}[1]{\textcolor[rgb]{0.31,0.60,0.02}{#1}}
\newcommand{\WarningTok}[1]{\textcolor[rgb]{0.56,0.35,0.01}{\textbf{\textit{#1}}}}
\usepackage{graphicx,grffile}
\makeatletter
\def\maxwidth{\ifdim\Gin@nat@width>\linewidth\linewidth\else\Gin@nat@width\fi}
\def\maxheight{\ifdim\Gin@nat@height>\textheight\textheight\else\Gin@nat@height\fi}
\makeatother
% Scale images if necessary, so that they will not overflow the page
% margins by default, and it is still possible to overwrite the defaults
% using explicit options in \includegraphics[width, height, ...]{}
\setkeys{Gin}{width=\maxwidth,height=\maxheight,keepaspectratio}
\setlength{\emergencystretch}{3em}  % prevent overfull lines
\providecommand{\tightlist}{%
  \setlength{\itemsep}{0pt}\setlength{\parskip}{0pt}}
\setcounter{secnumdepth}{0}
% Redefines (sub)paragraphs to behave more like sections
\ifx\paragraph\undefined\else
\let\oldparagraph\paragraph
\renewcommand{\paragraph}[1]{\oldparagraph{#1}\mbox{}}
\fi
\ifx\subparagraph\undefined\else
\let\oldsubparagraph\subparagraph
\renewcommand{\subparagraph}[1]{\oldsubparagraph{#1}\mbox{}}
\fi

% set default figure placement to htbp
\makeatletter
\def\fps@figure{htbp}
\makeatother


\title{Computer Lab 2}
\author{Dhyey Patel, Erik Anders}
\date{4/29/2020}

\begin{document}
\maketitle

\#\#1. Linear and polynomial regression (a) Determining the prior
distribution of the model parameters. Use the conjugate prior for the
linear regression model. Your task is to set the prior hyperparam- eters
μ 0 , Ω 0 , ν 0 and σ 0 2 to sensible values. Start with μ 0 = (−10,
100, −100) T , Ω 0 = 0.01 · I 3 , ν 0 = 4 and σ 0 2 = 1. Check if this
prior agrees with your prior opinions by simulating draws from the joint
prior of all parameters and for every draw compute the regression curve.
This gives a collection of regression curves, one for each draw from the
prior. Do the collection of curves look rea- sonable? If not, change the
prior hyperparameters until the collection of prior regression curves
agrees with your prior beliefs about the regression curve. {[}Hint: the
R package mvtnorm will be handy. And use your Inv-χ 2 simulator from Lab
1.{]}

\includegraphics{Computer-Lab-2_files/figure-latex/unnamed-chunk-1-1.pdf}

\begin{enumerate}
\def\labelenumi{(\alph{enumi})}
\setcounter{enumi}{1}
\tightlist
\item
  Write a program that simulates from the joint posterior distribution
  of β 0 , β 1 ,β 2 and σ 2 . Plot the marginal posteriors for each
  parameter as a histogram. Also produce another figure with a scatter
  plot of the temperature data and overlay a curve for the posterior
  median of the regression function f (time) = β 0 +β 1 ·time+β 2 ·time
  2 , computed for every value of time. Also overlay curves for the
  lower 2.5\% and upper 97.5\% posterior credible interval for f (time).
  That is, compute the 95\% equal tail posterior probability intervals
  for every value of time and then connect the lower and upper limits of
  the interval by curves. Does the interval bands contain most of the
  data points? Should they?
\end{enumerate}

\includegraphics{Computer-Lab-2_files/figure-latex/unnamed-chunk-2-1.pdf}
\includegraphics{Computer-Lab-2_files/figure-latex/unnamed-chunk-2-2.pdf}
\includegraphics{Computer-Lab-2_files/figure-latex/unnamed-chunk-2-3.pdf}
\includegraphics{Computer-Lab-2_files/figure-latex/unnamed-chunk-2-4.pdf}
\includegraphics{Computer-Lab-2_files/figure-latex/unnamed-chunk-2-5.pdf}

\#The interval bands contain most if the data points

\begin{enumerate}
\def\labelenumi{(\alph{enumi})}
\setcounter{enumi}{2}
\tightlist
\item
  It is of interest to locate the time with the highest expected
  temperature (that is, the time where f (time) is maximal). Let's call
  this value x̃. Use the simulations in b) to simulate from the posterior
  distribution of x̃. {[}Hint: the regression curve is a quadratic. You
  can find a simple formula for x̃ given β 0 , β 1 and β 2 .{]}
\end{enumerate}

\includegraphics{Computer-Lab-2_files/figure-latex/unnamed-chunk-3-1.pdf}

\begin{enumerate}
\def\labelenumi{(\alph{enumi})}
\setcounter{enumi}{3}
\tightlist
\item
  Say now that you want to estimate a polynomial model of order 7, but
  you suspect that higher order terms may not be needed, and you worry
  about over- fitting. Suggest a suitable prior that mitigates this
  potential problem. You do not need to compute the posterior, just
  write down your prior. {[}Hint: the task is to specify μ 0 and Ω 0 in
  a smart way.{]}
\end{enumerate}

\includegraphics{Computer-Lab-2_files/figure-latex/unnamed-chunk-4-1.pdf}

We want to reduce the impact of higher order terms in the polynomial
model of order 7. We have already seen in the first part that the
intercept, time and time squared term are necessary and so we try to
reduce the impact of the other higher order terms. We can do that by
taking the means and variances close to 0. Hence we take mu0 as very
small and omega0 as large values, as omega0 inverse is the multiplying
factor in the variance. Hence large values of omega0 for the higher
order terms will give very low variance.

\hypertarget{posterior-approximation-for-classification-with-logistic-regression}{%
\subsection{2. Posterior approximation for classification with logistic
regression}\label{posterior-approximation-for-classification-with-logistic-regression}}

\begin{enumerate}
\def\labelenumi{(\alph{enumi})}
\tightlist
\item
  Consider the logistic regression  exp x T β , Pr(y = 1\textbar{}x) =
  1 + exp (x T β) where y is the binary variable with y = 1 if the woman
  works and y = 0 if she does not. x is a 8-dimensional vector
  containing the eight features (including a one for the constant term
  that models the intercept). The goal is to approximate the posterior
  distribution of the 8-dim parameter vector β with a multivariate
  normal distribution   β\textbar{}y, X ∼ N β̃, J y −1 ( β̃) , where β̃
  is the posterior mode and J( β̃) = − ∂ 2 ln p(β\textbar{}y) \textbar{}
  β= β̃ is the observed Hes- ∂β∂β T 2 ln p(β\textbar{}y) sian evaluated
  at the posterior mode. Note that ∂ ∂β∂β is an 8×8 matrix with T 2 ln
  p(β\textbar{}y) second derivatives on the diagonal and
  cross-derivatives ∂ ∂β on the off- i ∂β j diagonal. It is actually not
  hard to compute this derivative by hand, but don't worry, we will let
  the computer do it numerically for you. Now, both β̃ and J( β̃) are
  computed by the optim function in R. See my code
  \url{https://github.com/}
  mattiasvillani/BayesLearnCourse/raw/master/Code/MainOptimizeSpam. zip
  where I have coded everything up for the spam prediction example (it
  also does probit regression, but that is not needed here). I want you
  to implement you own version of this. You can use my code as a
  template, but I want you 2to write your own file so that you
  understand every line of your code. Don't just copy my code. Use the
  prior β ∼ N (0, τ 2 I), with τ = 10. Your report should include your
  code as well as numerical values for β̃ and J y −1 ( β̃) for the
  WomenWork data. Compute an approximate 95\% credible in- terval for
  the variable NSmallChild. Would you say that this feature is an
  important determinant of the probability that a women works? {[}Hint:
  To verify that your results are reasonable, you can compare to you get
  by estimating the parameters using maximum likelihood: glmModel
  \textless{}- glm(Work \textasciitilde{} 0 + ., data = WomenWork,
  family = binomial).{]}
\end{enumerate}

\begin{verbatim}
## [1] "Optimal beta:"
\end{verbatim}

\begin{verbatim}
## [1]  0.62672884 -0.01979113  0.18021897  0.16756670 -0.14459669 -0.08206561
## [7] -1.35913317 -0.02468351
\end{verbatim}

\begin{verbatim}
##              [,1]          [,2]          [,3]          [,4]          [,5]
## [1,]  2.266022568  3.338861e-03 -6.545121e-02 -1.179140e-02  0.0457807243
## [2,]  0.003338861  2.528045e-04 -5.610225e-04 -3.125413e-05  0.0001414915
## [3,] -0.065451206 -5.610225e-04  6.218199e-03 -3.558209e-04  0.0018962893
## [4,] -0.011791404 -3.125413e-05 -3.558209e-04  4.351716e-03 -0.0142490853
## [5,]  0.045780724  1.414915e-04  1.896289e-03 -1.424909e-02  0.0555786706
## [6,] -0.030293450 -3.588562e-05 -3.240448e-06 -1.340888e-04 -0.0003299398
## [7,] -0.188748354  5.066847e-04 -6.134564e-03 -1.468951e-03  0.0032082535
## [8,] -0.098023929 -1.444223e-04  1.752732e-03  5.437105e-04  0.0005120144
##               [,6]          [,7]          [,8]
## [1,] -3.029345e-02 -0.1887483542 -0.0980239285
## [2,] -3.588562e-05  0.0005066847 -0.0001444223
## [3,] -3.240448e-06 -0.0061345645  0.0017527317
## [4,] -1.340888e-04 -0.0014689508  0.0005437105
## [5,] -3.299398e-04  0.0032082535  0.0005120144
## [6,]  7.184611e-04  0.0051841611  0.0010952903
## [7,]  5.184161e-03  0.1512621814  0.0067688739
## [8,]  1.095290e-03  0.0067688739  0.0199722657
\end{verbatim}

\includegraphics{Computer-Lab-2_files/figure-latex/unnamed-chunk-5-1.pdf}

Yes it is an important feature, it's absolute beta value is the highest
of the entire model. This means the value of the feature has a strong
impact on the result.

\begin{verbatim}
##    Constant  HusbandInc   EducYears    ExpYears   ExpYears2         Age 
##  0.64430363 -0.01977457  0.17988062  0.16751274 -0.14435946 -0.08234033 
## NSmallChild   NBigChild 
## -1.36250239 -0.02542986
\end{verbatim}

Looking at the estimated parameters using maximum likelihood we can see
that tey are very close the the ones we predicted.

\begin{enumerate}
\def\labelenumi{(\alph{enumi})}
\setcounter{enumi}{1}
\tightlist
\item
  Write a function that simulates from the predictive distribution of
  the response variable in a logistic regression. Use your normal
  approximation from 2(a). Use that function to simulate and plot the
  predictive distribution for the Work variable for a 40 year old woman,
  with two children (3 and 9 years old), 8 years of education, 10 years
  of experience. and a husband with an income of 10. {[}Hints: The R
  package mvtnorm will again be handy. Remember my discussion on how
  Bayesian prediction can be done by simulation.{]}
\end{enumerate}

\includegraphics{Computer-Lab-2_files/figure-latex/unnamed-chunk-7-1.pdf}

\begin{verbatim}
## [1] 22
\end{verbatim}

In our simulation 24\% had a job.

\begin{enumerate}
\def\labelenumi{(\alph{enumi})}
\setcounter{enumi}{2}
\tightlist
\item
  Now, consider 10 women which all have the same features as the woman
  in 2(b). Rewrite your function and plot the predictive distribution
  for the number of women, out of these 10, that are working. {[}Hint:
  Which distribution can be described as a sum of Bernoulli random
  variables?{]}
\end{enumerate}

\includegraphics{Computer-Lab-2_files/figure-latex/unnamed-chunk-8-1.pdf}

\#\#Code Appendix

\begin{Shaded}
\begin{Highlighting}[]
\NormalTok{knitr}\OperatorTok{::}\NormalTok{opts_chunk}\OperatorTok{$}\KeywordTok{set}\NormalTok{(}\DataTypeTok{echo =} \OtherTok{TRUE}\NormalTok{)}
\CommentTok{#1.}
\CommentTok{#a).}
\NormalTok{n <-}\StringTok{ }\DecValTok{1}
\NormalTok{data <-}\StringTok{ }\KeywordTok{read.csv}\NormalTok{(}\StringTok{"/home/erik/Documents/SML/Semester 2/Bayesian Learning/Labs/Lab2/TempLinkoping.txt"}\NormalTok{, }\DataTypeTok{sep=}\StringTok{""}\NormalTok{)}
\NormalTok{X <-}\StringTok{ }\KeywordTok{cbind}\NormalTok{(}\DecValTok{1}\NormalTok{,data}\OperatorTok{$}\NormalTok{time,data}\OperatorTok{$}\NormalTok{time}\OperatorTok{^}\DecValTok{2}\NormalTok{)}

\NormalTok{mu_}\DecValTok{0}\NormalTok{ <-}\StringTok{ }\KeywordTok{c}\NormalTok{(}\OperatorTok{-}\DecValTok{10}\NormalTok{,}\DecValTok{130}\NormalTok{,}\OperatorTok{-}\DecValTok{130}\NormalTok{)}
\NormalTok{sigma2_}\DecValTok{0}\NormalTok{ <-}\StringTok{ }\DecValTok{1}
\NormalTok{omega_}\DecValTok{0}\NormalTok{ <-}\StringTok{ }\FloatTok{0.1}\OperatorTok{*}\KeywordTok{diag}\NormalTok{(}\DecValTok{3}\NormalTok{)}
\NormalTok{vu_}\DecValTok{0}\NormalTok{ <-}\StringTok{ }\DecValTok{4}

\KeywordTok{library}\NormalTok{(MASS)}
\KeywordTok{library}\NormalTok{(mvtnorm)}

\ControlFlowTok{for}\NormalTok{ (i }\ControlFlowTok{in} \DecValTok{1}\OperatorTok{:}\DecValTok{100}\NormalTok{) \{}
\NormalTok{  sigma2 <-}\StringTok{ }\NormalTok{(sigma2_}\DecValTok{0}\OperatorTok{*}\NormalTok{vu_}\DecValTok{0}\NormalTok{)}\OperatorTok{/}\KeywordTok{rchisq}\NormalTok{(}\DataTypeTok{n =} \DecValTok{1}\NormalTok{, }\DataTypeTok{df =}\NormalTok{ vu_}\DecValTok{0}\NormalTok{)}
  \CommentTok{#beta_sigma2 <- mvrnorm(n = n, mu = mu_0, Sigma = sigma2*solve(omega_0))}
\NormalTok{  beta_sigma2 <-}\StringTok{ }\KeywordTok{rmvnorm}\NormalTok{(}\DataTypeTok{n =} \DecValTok{1}\NormalTok{, }\DataTypeTok{mean =}\NormalTok{ mu_}\DecValTok{0}\NormalTok{, }\DataTypeTok{sigma =}\NormalTok{ sigma2}\OperatorTok{*}\KeywordTok{solve}\NormalTok{(omega_}\DecValTok{0}\NormalTok{))}
  
  \CommentTok{#temp <- X%*%beta_sigma2}
\NormalTok{  E <-}\StringTok{ }\KeywordTok{rnorm}\NormalTok{(}\DataTypeTok{n =} \DecValTok{1}\NormalTok{, }\DataTypeTok{mean =} \DecValTok{0}\NormalTok{,}\DataTypeTok{sd =} \KeywordTok{sqrt}\NormalTok{(sigma2))}
\NormalTok{  temp <-}\StringTok{ }\NormalTok{X}\OperatorTok\KeywordTok{t}\NormalTok{(beta_sigma2) }\OperatorTok{+}\StringTok{ }\NormalTok{E}
  
  \ControlFlowTok{if}\NormalTok{ (i }\OperatorTok{==}\StringTok{ }\DecValTok{1}\NormalTok{) \{}
    \KeywordTok{plot}\NormalTok{(temp, }\DataTypeTok{type =} \StringTok{"l"}\NormalTok{, }\DataTypeTok{ylim =} \KeywordTok{c}\NormalTok{(}\OperatorTok{-}\DecValTok{50}\NormalTok{,}\DecValTok{50}\NormalTok{))}
\NormalTok{  \}}
  \ControlFlowTok{else}\NormalTok{\{}
    \KeywordTok{lines}\NormalTok{(temp, }\DataTypeTok{type =} \StringTok{"l"}\NormalTok{)}
\NormalTok{  \}}
\NormalTok{\}}
\CommentTok{#b).}
\NormalTok{X <-}\StringTok{ }\KeywordTok{cbind}\NormalTok{(}\DecValTok{1}\NormalTok{,data}\OperatorTok{$}\NormalTok{time,data}\OperatorTok{$}\NormalTok{time}\OperatorTok{^}\DecValTok{2}\NormalTok{)}
\NormalTok{y <-}\StringTok{ }\NormalTok{data}\OperatorTok{$}\NormalTok{temp}
\NormalTok{n <-}\StringTok{ }\KeywordTok{length}\NormalTok{(data}\OperatorTok{$}\NormalTok{time)}

\NormalTok{beta_hat <-}\StringTok{ }\KeywordTok{solve}\NormalTok{(}\KeywordTok{t}\NormalTok{(X) }\OperatorTok\StringTok{ }\NormalTok{X) }\OperatorTok\StringTok{ }\NormalTok{(}\KeywordTok{t}\NormalTok{(X) }\OperatorTok\StringTok{ }\NormalTok{y)}
\NormalTok{mu_n <-}\StringTok{ }\KeywordTok{solve}\NormalTok{(}\KeywordTok{t}\NormalTok{(X)}\OperatorTok\NormalTok{X}\OperatorTok{+}\NormalTok{omega_}\DecValTok{0}\NormalTok{)}\OperatorTok\NormalTok{(}\KeywordTok{t}\NormalTok{(X)}\OperatorTok\NormalTok{X}\OperatorTok\NormalTok{beta_hat}\OperatorTok{+}\NormalTok{omega_}\DecValTok{0}\OperatorTok\NormalTok{mu_}\DecValTok{0}\NormalTok{)}
\NormalTok{omega_n <-}\StringTok{ }\KeywordTok{t}\NormalTok{(X)}\OperatorTok\NormalTok{X}\OperatorTok{+}\NormalTok{omega_}\DecValTok{0}
\NormalTok{vu_n <-}\StringTok{ }\NormalTok{vu_}\DecValTok{0} \OperatorTok{+}\StringTok{ }\NormalTok{n }\CommentTok{#what n?}
\NormalTok{vu_nsigma2 <-}\StringTok{ }\NormalTok{vu_}\DecValTok{0}\OperatorTok{*}\NormalTok{sigma2_}\DecValTok{0} \OperatorTok{+}\StringTok{ }\NormalTok{(}\KeywordTok{t}\NormalTok{(y)}\OperatorTok\NormalTok{y}\OperatorTok{+}\KeywordTok{t}\NormalTok{(mu_}\DecValTok{0}\NormalTok{)}\OperatorTok\NormalTok{omega_}\DecValTok{0}\OperatorTok\NormalTok{mu_}\DecValTok{0}\OperatorTok{-}\KeywordTok{t}\NormalTok{(mu_n)}\OperatorTok\NormalTok{omega_n}\OperatorTok\NormalTok{mu_n)}

\NormalTok{sigma2_n <-}\StringTok{ }\KeywordTok{var}\NormalTok{(data}\OperatorTok{$}\NormalTok{temp)}
\NormalTok{sigma2_i <-}\StringTok{ }\KeywordTok{c}\NormalTok{()}
\NormalTok{beta_}\DecValTok{0}\NormalTok{ <-}\StringTok{ }\KeywordTok{c}\NormalTok{()}
\NormalTok{beta_}\DecValTok{1}\NormalTok{ <-}\StringTok{ }\KeywordTok{c}\NormalTok{()}
\NormalTok{beta_}\DecValTok{2}\NormalTok{ <-}\StringTok{ }\KeywordTok{c}\NormalTok{()}
\NormalTok{f_time <-}\StringTok{ }\KeywordTok{c}\NormalTok{()}
\NormalTok{f_median <-}\StringTok{ }\KeywordTok{c}\NormalTok{()}

\ControlFlowTok{for}\NormalTok{ (i }\ControlFlowTok{in} \DecValTok{1}\OperatorTok{:}\DecValTok{100}\NormalTok{) \{}
\NormalTok{  sigma2 <-}\StringTok{ }\NormalTok{(vu_nsigma2)}\OperatorTok{/}\KeywordTok{rchisq}\NormalTok{(}\DataTypeTok{n =} \DecValTok{1}\NormalTok{, }\DataTypeTok{df =}\NormalTok{ vu_n)}
\NormalTok{  beta_sigma2 <-}\StringTok{ }\KeywordTok{rmvnorm}\NormalTok{(}\DataTypeTok{n =} \DecValTok{1}\NormalTok{, }\DataTypeTok{mean =}\NormalTok{ mu_n, }\DataTypeTok{sigma =}\NormalTok{ sigma2[}\DecValTok{1}\NormalTok{,}\DecValTok{1}\NormalTok{]}\OperatorTok{*}\KeywordTok{solve}\NormalTok{(omega_n))}
  
  \CommentTok{#temp <- X%*%beta_sigma2}
\NormalTok{  E <-}\StringTok{ }\KeywordTok{rnorm}\NormalTok{(}\DataTypeTok{n =} \DecValTok{1}\NormalTok{, }\DataTypeTok{mean =} \DecValTok{0}\NormalTok{,}\DataTypeTok{sd =} \KeywordTok{sqrt}\NormalTok{(sigma2))}
\NormalTok{  temp <-}\StringTok{ }\NormalTok{X}\OperatorTok\KeywordTok{t}\NormalTok{(beta_sigma2) }\OperatorTok{+}\StringTok{ }\NormalTok{E}
\NormalTok{  f_time <-}\StringTok{ }\KeywordTok{cbind}\NormalTok{(f_time, }\KeywordTok{as.vector}\NormalTok{(temp))}
  
  \ControlFlowTok{if}\NormalTok{ (i }\OperatorTok{==}\StringTok{ }\DecValTok{1}\NormalTok{) \{}
    \KeywordTok{plot}\NormalTok{(temp, }\DataTypeTok{type =} \StringTok{"l"}\NormalTok{, }\DataTypeTok{ylim =} \KeywordTok{c}\NormalTok{(}\OperatorTok{-}\DecValTok{50}\NormalTok{,}\DecValTok{50}\NormalTok{))}
\NormalTok{  \}}
  \ControlFlowTok{else}\NormalTok{\{}
    \KeywordTok{lines}\NormalTok{(temp, }\DataTypeTok{type =} \StringTok{"l"}\NormalTok{)}
\NormalTok{  \}}
\NormalTok{  beta_}\DecValTok{0}\NormalTok{[i] <-}\StringTok{ }\NormalTok{beta_sigma2[}\DecValTok{1}\NormalTok{,}\DecValTok{1}\NormalTok{]}
\NormalTok{  beta_}\DecValTok{1}\NormalTok{[i] <-}\StringTok{ }\NormalTok{beta_sigma2[}\DecValTok{1}\NormalTok{,}\DecValTok{2}\NormalTok{]}
\NormalTok{  beta_}\DecValTok{2}\NormalTok{[i] <-}\StringTok{ }\NormalTok{beta_sigma2[}\DecValTok{1}\NormalTok{,}\DecValTok{3}\NormalTok{]}
\NormalTok{  sigma2_i[i] <-}\StringTok{ }\NormalTok{sigma2 }
\NormalTok{\}}
\KeywordTok{points}\NormalTok{(f_median)}


\KeywordTok{hist}\NormalTok{(beta_}\DecValTok{0}\NormalTok{, }\DataTypeTok{main =} \StringTok{"marginal posterior beta_0"}\NormalTok{)}
\KeywordTok{hist}\NormalTok{(beta_}\DecValTok{1}\NormalTok{, }\DataTypeTok{main =} \StringTok{"marginal posterior beta_1"}\NormalTok{)}
\KeywordTok{hist}\NormalTok{(beta_}\DecValTok{2}\NormalTok{, }\DataTypeTok{main =} \StringTok{"marginal posterior beta_2"}\NormalTok{)}
\KeywordTok{hist}\NormalTok{(sigma2_i, }\DataTypeTok{main =} \StringTok{"marginal posterior sigma²")}

\StringTok{for (i in 1:365) \{}
\StringTok{  f_median <- median(f_time[i,])}
\StringTok{\}}

\StringTok{#95% equal tail credible interval}
\StringTok{G <- f_time[,1]}

\StringTok{#c).}
\StringTok{x_max <- -beta_1/(2*beta_2)}
\StringTok{plot(density(x_max))}
\StringTok{abline(v=data$time[which.max(f_median)], col = "}\NormalTok{red}\StringTok{")}
\StringTok{X <- cbind(1,data$time)}
\StringTok{for (i in 2:7) \{}
\StringTok{  X <- cbind(X,data$time^i)}
\StringTok{\}}

\StringTok{mu_0 <- c(-10,130,-130, rep(0.01,5))}
\StringTok{sigma2_0 <- 1}
\StringTok{omega_0 <- c(rep(0.02,3),rep(10,5))*diag(8)}
\StringTok{vu_0 <- 4}

\StringTok{for (i in 1:100) \{}
\StringTok{  sigma2 <- (sigma2_0*vu_0)/rchisq(n = 1, df = vu_0)}
\StringTok{  #beta_sigma2 <- mvrnorm(n = n, mu = mu_0, Sigma = sigma2*solve(omega_0))}
\StringTok{  beta_sigma2 <- rmvnorm(n = 1, mean = mu_0, sigma = sigma2*solve(omega_0))}
\StringTok{  }
\StringTok{  #temp <- X%*%beta_sigma2}
\StringTok{  E <- rnorm(n = 1, mean = 0,sd = sqrt(sigma2))}
\StringTok{  temp <- X%*%t(beta_sigma2) + E}
\StringTok{  }
\StringTok{  if (i == 1) \{}
\StringTok{    plot(temp, type = "}\NormalTok{l}\StringTok{", ylim = c(-50,50))}
\StringTok{  \}}
\StringTok{  else\{}
\StringTok{    lines(temp, type = "}\NormalTok{l}\StringTok{")}
\StringTok{  \}}
\StringTok{\}}

\StringTok{#2.}
\StringTok{#a).}
\StringTok{data <- read.table("}\OperatorTok{/}\NormalTok{home}\OperatorTok{/}\NormalTok{erik}\OperatorTok{/}\NormalTok{Documents}\OperatorTok{/}\NormalTok{SML}\OperatorTok{/}\NormalTok{Semester }\DecValTok{2}\OperatorTok{/}\NormalTok{Bayesian Learning}\OperatorTok{/}\NormalTok{Labs}\OperatorTok{/}\NormalTok{Lab2}\OperatorTok{/}\NormalTok{WomenWork.dat}\StringTok{", header = TRUE)}
\StringTok{n <- dim(data)[1]}

\StringTok{x <- as.matrix(data[,2:9])}
\StringTok{y <- as.vector(data[,1])}

\StringTok{nPara <- dim(x)[2];}

\StringTok{# Setting up the prior}
\StringTok{mu <- as.vector(rep(0,nPara))}
\StringTok{tau <- 10}
\StringTok{sigma <- tau^2*diag(nPara)}


\StringTok{LogPostLogistic <- function(betaVect,y,X,mu,Sigma)\{}
\StringTok{  }
\StringTok{  nPara <- length(betaVect);}
\StringTok{  linPred <- X%*%betaVect;}
\StringTok{  }
\StringTok{  # evaluating the log-likelihood                                    }
\StringTok{  logLik <- sum( linPred*y -log(1 + exp(linPred)));}
\StringTok{  if (abs(logLik) == Inf) logLik = -20000; # Likelihood is not finite, stear the optimizer away from here!}
\StringTok{  }
\StringTok{  # evaluating the prior}
\StringTok{  logPrior <- dmvnorm(betaVect, matrix(0,nPara,1), Sigma, log=TRUE);}
\StringTok{  }
\StringTok{  # add the log prior and log-likelihood together to get log posterior}
\StringTok{  return(logLik + logPrior)}
\StringTok{\}}


\StringTok{#initVal <- as.vector(rmvnorm(n = 1, mean = rep(0,dim(x)[2]), sigma = tau^2*diag(dim(x)[2])))}
\StringTok{initVal <- as.vector(rep(0,dim(x)[2]))}
\StringTok{posterior <- optim(initVal,LogPostLogistic,gr=NULL,y,x,mu,sigma,method=c("}\NormalTok{BFGS}\StringTok{"),control=list(fnscale=-1),hessian=TRUE)}
\StringTok{posterior_mode <- posterior$par}
\StringTok{hessian_matrix <- posterior$hessian}
\StringTok{hessian_sigma <- -solve(posterior$hessian)}

\StringTok{print("}\NormalTok{Optimal beta}\OperatorTok{:}\StringTok{")}
\StringTok{print(posterior_mode)}

\StringTok{#}
\StringTok{print(hessian_sigma)}

\StringTok{#95% equal tail credible interval}
\StringTok{for (i in 1:1000) \{}
\StringTok{  G[i] <- rnorm(1, posterior_mode[7], hessian_sigma[7,7])}
\StringTok{\}}

\StringTok{G_sort <- sort(G)}
\StringTok{#G_cut <- G_sort[(length(G_sort)*0.05):(length(G_sort)*0.95-1)]}
\StringTok{plot(G_sort)}
\StringTok{abline(v=(length(G_sort)*0.05), col = "}\NormalTok{red}\StringTok{")}
\StringTok{abline(v=length(G_sort)*0.95-1, col = "}\NormalTok{red}\StringTok{")}
\StringTok{#Yes it is an important feature, it's absolute beta value is the highest of the entire model.}
\StringTok{#This means the value of the feature has a strong impact on the result.}

\StringTok{glmModel <- glm(Work ~ 0 + ., data = data, family = binomial)}
\StringTok{glmModel$coefficients}

\StringTok{#Looking at the estimated parameters using maximum likelihood we can see that tey are very close the the ones we predicted.}

\StringTok{#b).}
\StringTok{X <- c(1,10,8,10,1,40,1,1)}
\StringTok{result <- c()}
\StringTok{for (i in 1:100) \{}
\StringTok{  beta_posterior <- rmvnorm(1, posterior_mode, hessian_sigma)}
\StringTok{  y <- exp(t(X) %*% t(beta_posterior))/(1+(exp(t(X) %*% t(beta_posterior))))}
\StringTok{  result[i] <- rbinom(1,1, y)}
\StringTok{\}}
\StringTok{plot(density(result), main = "}\NormalTok{p. distribution }\ControlFlowTok{for}\NormalTok{ Work}\StringTok{")}

\StringTok{sum(result==1)}
\StringTok{#c).}
\StringTok{X <- c(1,10,8,10,1,40,1,1)}
\StringTok{result <- c()}
\StringTok{for (i in 1:100) \{}
\StringTok{  beta_posterior <- rmvnorm(1, posterior_mode, hessian_sigma)}
\StringTok{  y <- exp(t(X) %*% t(beta_posterior))/(1+(exp(t(X) %*% t(beta_posterior))))}
\StringTok{  result[i] <- rbinom(1,10, y)}
\StringTok{\}}
\StringTok{plot(density(result), main = "}\NormalTok{working women}\StringTok{" , xlab = "}\NormalTok{working women}\StringTok{")}
\end{Highlighting}
\end{Shaded}

\end{document}
